%%%%%%%%%%%%%%%%%%%%%%%%%%%%%%%%%%%%%%%%%
% Tutorial
% LaTeX Template
% Version 1.0 (09/27/17)
%
% Author:
% Ben Roose (ben.roose@wichita.edu)
%
% Original template author:
% Adam Glesser (adamglesser@gmail.com)
% www.LaTeXTemplates.com
%
% License:
% CC BY-NC-SA 3.0 (http://creativecommons.org/licenses/by-nc-sa/3.0/)
%
%%%%%%%%%%%%%%%%%%%%%%%%%%%%%%%%%%%%%%%%%

\documentclass[12pt]{article}

\usepackage{graphicx} % Allow import of images
\usepackage{subcaption} % Required for side-by-side sub figures
\graphicspath{ {images/} } % Relative path to images directory
\usepackage[margin=1in]{geometry} % Required to make the margins smaller to fit more content on each page
\usepackage[linkcolor=blue]{hyperref} % Required to create hyperlinks to questions from elsewhere in the document
\hypersetup{pdfborder={0 0 0}, colorlinks=true, urlcolor=blue} % Specify a color for hyperlinks
\usepackage{todonotes} % Required for the boxes that questions appear in
\usepackage{tocloft} % Required to give customize the table of contents to display questions
\usepackage{microtype} % Slightly tweak font spacing for aesthetics
\usepackage{palatino} % Use the Palatino font

\setlength\parindent{0pt} % Removes all indentation from paragraphs

% Create and define the list of questions
\newlistof{questions}{faq}{\large FAQ for grader access to cslab Linux environment}
% This creates a new table of contents-like environment that will output a file with extension .faq
\setlength\cftbeforefaqtitleskip{3em} % Adjusts the vertical space between the title and subtitle
\setlength\cftafterfaqtitleskip{1em} % Adjusts the vertical space between the subtitle and the first question
\setlength\cftparskip{.3em} % Adjusts the vertical space between questions in the list of questions

% Create the command used for questions
\newcommand{\question}[1] % This is what you will use to create a new question
{
\refstepcounter{questions} % Increases the questions counter, this can be referenced anywhere with \thequestions
%\hfill
\goodbreak
\par\noindent % Creates a new unindented paragraph
\phantomsection % Needed for hyperref compatibility with the \addcontensline command
\addcontentsline{faq}{questions}{#1} % Adds the question to the list of questions
\todo[inline, color=green!40]{\textbf{#1}} % Uses the todonotes package to create a fancy box to put the question
%\vspace{0.5em} % White space after the question before the start of the answer
}

% Uncomment the line below to get rid of the trailing dots in the table of contents
%\renewcommand{\cftdot}{}

% Uncomment the two lines below to get rid of the numbers in the table of contents
%\let\Contentsline\contentsline
%\renewcommand\contentsline[3]{\Contentsline{#1}{#2}{}}

\begin{document}

%----------------------------------------------------------------------------------------
%	TITLE AND LIST OF QUESTIONS
%----------------------------------------------------------------------------------------

\begin{center}
\Huge{\bf \emph{EECS Tutorial: Grader access to cslab Linux Environment}} % Main title
\end{center}

\listofquestions % This prints the subtitle and a list of all of your questions
\bigskip % Create a gap between list and first question

\newpage % Comment this if you would like your questions and answers to start immediately after table of questions

%----------------------------------------------------------------------------------------
%	QUESTIONS AND ANSWERS
%----------------------------------------------------------------------------------------
\begin{flushleft}

\question{How do I initially log into the cslab Linux environment?}\label{guac_login}

\begin{itemize}
  \item To access the cslab Linux environment, \href{https://cslab-gateway.cs.wichita.edu/}{cslab-gateway.cs.wichita.edu}, follow the steps as found in the separate \textit{EECS Tutorial: cslab Linux Environment} document using your myWSU ID (with lowercase letters for username) and your myWSU password.
  \item If you cannot access the cslab environment, ensure you have changed your myWSU password within the last three months and can access the myWSU main website \href{https://mywsu.wichita.edu}{mywsu.wichita.edu}. If you cannot access the myWSU website, then you will need to change your password before gaining access into the cslab environment.
  \item If you can access myWSU but cannot access the cslab environment, please contact the EECS systems administrator, \href{mailto:ben.roose@wichita.edu}{ben.roose@wichita.edu}.
\end{itemize}

%------------------------------------------------

\question{How do I set up an SSH private/public key for accessing a grader account?}\label{ssh_key_gen}

\begin{enumerate}
  \item Once logged into the cslab environment, open an SSH terminal session.
  \item Before you generate an SSH key, you can check to see if you have any existing SSH keys in your user account by typing \break
  \texttt{ls $\sim$/.ssh}
  \item Check the directory listing to see if you already have a public SSH key. By default, the filenames of the public keys are one of the following:
\begin{verbatim}
    id_rsa.pub
    id_dsa.pub
    id_ecdsa.pub
    id_ed25519.pub
\end{verbatim}
  \item If you do not have an SSH key or you wish to generate a new one, type \break
\texttt{ssh-keygen -t rsa -b 4096 -C "your\_mywsu\_id-your\_last\_name"}

\item Follow the prompts in the command-line as your new SSH key is generated:
  
      \texttt{Generating public/private rsa key pair.} \break
      \texttt{Enter file in which to save the key (../id\_rsa):} Press Enter \break
      \texttt{Enter passphrase:} Enter a passphrase you will remember! \break
      \texttt{Your identification has been saved in .../.ssh/id\_rsa.} \break
      \texttt{Your public key has been saved in .../.ssh/id\_rsa.pub.} \break
      \texttt{The key fingerprint is:} \break
      \texttt{SHA256: [fingerprint and randomart image]}
      
      \textbf{NOTE: Unless you know how to use custom SSH key files, always use the default filename and path for saving your SSH key.} \break
      \textbf{You are required to use a passphrase for your SSH key to access grader accounts.}
    
  \item Download your SSH public key file to your local computer by typing \break
    \texttt{guacget $\sim$/.ssh/id\_rsa.pub}
  \item If the \texttt{id\_rsa.pub} file does not download to your local computer via the web-bowser, then you may need to disconnect from the terminal session by pressing CTRL+D and clicking the ``reconnect'' button.
  \item Attach the downloaded \texttt{id\_rsa.pub} file to an email and send it both to the instructor of the class and to the EECS systems administrator (\href{mailto:ben.roose@wichita.edu}{ben.roose@wichita.edu}) with the following additional details in the email body:
\begin{verbatim}
myWSU_ID: your_myWSU_ID, i.e. a123b456
Instructor: instructor_of_the_class i.e. Adam Sweeney
Grader Account: grader_account_as_assigned_to_you, i.e. cs211a
\end{verbatim}
  \item If you are a grader for more than one programming class, you will only need to generate one SSH key but you will need to append additional \texttt{Instructor} and \texttt{Grader Account} lines for each class you will be grading.
 \item Once your SSH public key has been added to the \texttt{authorized\_keys} for the grader account by the EECS systems adminstrator, you will then have access into it.
\end{enumerate}

%------------------------------------------------

\question{How do I access a grader account for grading assignments?}\label{grader_access}
\begin{enumerate}
  \item Once logged into the cslab environment, open either an RDP desktop session or an SSH terminal session.
  \item If you are connected to an RDP desktop session, open a terminal emulator, i.e. \textit{LXTerminal} or \textit{Terminator}.
  \item To access your assigned grader account type \break
\texttt{ssh grader\_account@localhost} \break
(where \texttt{grader\_account} is the class account, i.e. \texttt{cs211a}).
\item Type in your recently created ``passphrase'' when prompted. \break
  If you are prompted for a ``password'' not a ``passphrase'', then there is something wrong with your SSH key. Double-check you generated an SSH key and contact the EECS systems administrator if you are not able to log into the grader account.
  \item Once logged into the grader account, you can use standard Linux command-line tools to access, view, and compile student assignments. \break
    For example: \texttt{ls, cd, less, vi, emacs, gcc, g++}.
  \item You will find each submitted student assignment in the directory \break
    \texttt{$\sim$/handin/assign/\textit{assignment\_\#}/\textit{student\_myWSU\_ID}/}
  \item If a student submitted an assignment late, you will find it in the directory \break
    \texttt{$\sim$/handin/assign/\textit{assignment\_\#}.late/\textit{student\_myWSU\_ID}/}
  \item To allow for graphical (GUI) applications, such as \textit{geany} or \textit{atom} to be used, add the \texttt{-X} option flag to the \texttt{ssh} command, i.e. \break
\texttt{ssh -X cs211a@localhost}
  \item To use a graphical application type the application's name followed by an ampersand (\&) in the terminal emulator. For example: \texttt{geany \&} \break
    
\textbf{NOTE: ssh -X option for graphical applications can ONLY be used within an RDP desktop session.}
\end{enumerate}
\end{flushleft}


%------------------------------------------------

\end{document}
